\documentclass{article}\usepackage[]{graphicx}\usepackage[]{color}
%% maxwidth is the original width if it is less than linewidth
%% otherwise use linewidth (to make sure the graphics do not exceed the margin)
\makeatletter
\def\maxwidth{ %
  \ifdim\Gin@nat@width>\linewidth
    \linewidth
  \else
    \Gin@nat@width
  \fi
}
\makeatother

\definecolor{fgcolor}{rgb}{0.345, 0.345, 0.345}
\newcommand{\hlnum}[1]{\textcolor[rgb]{0.686,0.059,0.569}{#1}}%
\newcommand{\hlstr}[1]{\textcolor[rgb]{0.192,0.494,0.8}{#1}}%
\newcommand{\hlcom}[1]{\textcolor[rgb]{0.678,0.584,0.686}{\textit{#1}}}%
\newcommand{\hlopt}[1]{\textcolor[rgb]{0,0,0}{#1}}%
\newcommand{\hlstd}[1]{\textcolor[rgb]{0.345,0.345,0.345}{#1}}%
\newcommand{\hlkwa}[1]{\textcolor[rgb]{0.161,0.373,0.58}{\textbf{#1}}}%
\newcommand{\hlkwb}[1]{\textcolor[rgb]{0.69,0.353,0.396}{#1}}%
\newcommand{\hlkwc}[1]{\textcolor[rgb]{0.333,0.667,0.333}{#1}}%
\newcommand{\hlkwd}[1]{\textcolor[rgb]{0.737,0.353,0.396}{\textbf{#1}}}%
\let\hlipl\hlkwb

\usepackage{framed}
\makeatletter
\newenvironment{kframe}{%
 \def\at@end@of@kframe{}%
 \ifinner\ifhmode%
  \def\at@end@of@kframe{\end{minipage}}%
  \begin{minipage}{\columnwidth}%
 \fi\fi%
 \def\FrameCommand##1{\hskip\@totalleftmargin \hskip-\fboxsep
 \colorbox{shadecolor}{##1}\hskip-\fboxsep
     % There is no \\@totalrightmargin, so:
     \hskip-\linewidth \hskip-\@totalleftmargin \hskip\columnwidth}%
 \MakeFramed {\advance\hsize-\width
   \@totalleftmargin\z@ \linewidth\hsize
   \@setminipage}}%
 {\par\unskip\endMakeFramed%
 \at@end@of@kframe}
\makeatother

\definecolor{shadecolor}{rgb}{.97, .97, .97}
\definecolor{messagecolor}{rgb}{0, 0, 0}
\definecolor{warningcolor}{rgb}{1, 0, 1}
\definecolor{errorcolor}{rgb}{1, 0, 0}
\newenvironment{knitrout}{}{} % an empty environment to be redefined in TeX

\usepackage{alltt}

\title{CCEVI of Skew Copula}
\author{Vincenzo Coia}

\usepackage{amsmath}
\usepackage{amsthm}
\usepackage{amssymb}

\newtheorem{theorem}{Theorem}

\def\Chat{{\widehat C}}
\def\Sk{\mathrm{\textbf{Skew}}}
\def\a{\alpha}
\IfFileExists{upquote.sty}{\usepackage{upquote}}{}
\begin{document}

\maketitle

\section{Skew Copulas}
\label{sec:ccevi-skew}

Bivariate copulas can be transformed to have a skew, adding permutation asymmetry to the copula. 
A skew transformation to a copula $C$ for parameter $\alpha\in[0,1]$ is
\begin{equation}
\label{eq:cop_skew}
C_{\Sk}(u,v)=C(u,v^{1-\alpha})v^{\alpha}
\end{equation}
for $(u,v)\in[0,1]^2$, where larger values of $\alpha$ correspond to a larger skew.

\begin{theorem}
\label{prop:ccevi_skew}
Suppose a bivariate copula $C$ and its reflection $\Chat$ have CCEVI's $\xi_C>0$ and $\xi_{\Chat}>0$, respectively. Then the skew copula defined in \eqref{eq:cop_skew} for $\alpha\in[0,1]$ has a CCEVI of 1 (or $\xi_C$ if $\alpha=0$), and the reflection skew copula has CCEVI
\begin{equation}
\label{eq:ccevi_skew_refl}
\frac{1}{\a + (1-\a)(1/\xi_{\Chat})}.
\end{equation}
\end{theorem}

See Appendix~\ref{app:pfs-skew} for a proof.

Notice that the reciprocal CCEVI of the skew reflection copula in~\eqref{eq:ccevi_skew_refl} is an interpolation of 1 and the reciprocal CCEVI of the original skew copula.


\subsection{Proof of CCEVI of Skew Copula Class}
\label{app:pfs-skew}

Here, we present a proof of Proposition~\ref{prop:ccevi_skew}, first for the (non-reflected) skew copula, then for the reflected. 

First, recall the expansion
\begin{equation}
(1-t^{-1})^{\a} 
    = 1 - \a t^{-1} + \frac{\a(\a-1)}{2!}t^{-2} + O(t^{-3})
    = 1 - t^{-1}O(1),
\end{equation}
valid for $t$ near infinity.
Now, differentiate the skew copula with respect to the first argument to obtain the copula conditional distribution. Then substitute $v=1-1/t$ and send $t\rightarrow\infty$:
\begin{align}
1 - C_{\Sk, 2|1}(v|u) 
    &=    1 - C_{2|1}(v^{1-\a} | u) v^{\a} \\
    &=    1 - C_{2|1}(1-t^{-1}O(1)|u)
                       (1-t^{-1}O(1)) \\
    &\sim 1 - (1 - \ell(t)t^{-1/\xi_C})(1-t^{-1}) \\
    &=    t^{-1} + O(t^{-1}),
\end{align}
where $\ell$ is slowly varying at infinity. 
The CCEVI of the skew copula is therefore 1.

The reflected skew copula is
\begin{equation}
\Chat_{\Sk}(u,v) =
    u + v - 1 + C(1-u, (1-v)^{1-\a}) (1-v)^{\a}.
\end{equation}
The trick is to write the copula conditional distribution in terms of the reflected original copula, again substituting $v=1-1/t$ and sending $t\rightarrow\infty$:
\begin{align}
1 - \Chat_{\Sk,2|1}(v|u)
    &=    (1-v)^{\a} C_{2|1}((1-v)^{1-\a}|1-u)  \\
    &=    t^{-\a} C_{2|1}(t^{-(1-\a)}|1-u) \\
    &=    t^{-\a} \left(1 - \Chat_{2|1} (1 - t^{-(1-\a)} | u) \right) \\
    &\sim t^{-\a} \hat{\ell}(t) \left(t^{1-\a}\right) ^ {-1/\xi_{\Chat}(u)} \\
    &=    \hat{\ell}(t) t^{-1/\xi_{\widehat{\Sk}}(u)},
\end{align}
where $\hat{\ell}$ is slowly varying at infinity, and 
\begin{equation}
\xi_{\widehat{\Sk}}(u)
    = \frac{1}{\a + (1-\a)(1/\xi_{\Chat}(u))}
\end{equation}
is the CCEVI of the reflected skew copula.


\end{document}
